%----------------------------------------------------------------------------------------
%   
%   SUPER COOL RESUME TEMPLATE
%
%   Resume Template by Spencer Elkington, for the
%   purposes of the U Professional Education Program.
%
%   Welcome to your new resume! I'll be annotating how to apply the concepts
%   from our resume workshop below.
%   
%   Feel free to delete this top bit if you ever post the source online - take
%   credit for all of this amazing work, if you really want!
%
%   SOME BASICS:
%   
%   Anything with a '%' (percent sign) before it is COMMENTED CODE. This will keep
%   it from appearing in the final LaTeX render. I use it for two reasons:
%
%     1. Making comments like these.
%     2. Omitting things from my resume that I don't want NOW, but may want LATER.
%
%   Things will come and go from your resume - some experiences you have may not
%   feel applicable to a job you're currently applying for - DO NOT DELETE THEM.
%   You never know when experience may become relevant again!
%
%   COMMENT IT OUT USING '%', because then you can whip out your experience again
%   when you need it!
%
%   Also, sometimes it's a cool confidence booster to un-comment everything and see
%   all the neat stuff you've done so far in your professional life :)
%
%----------------------------------------------------------------------------------------


%----------------------------------------------------------------------------------------
%
%  PACKAGES AND OTHER DOCUMENT CONFIGURATIONS
%
%   These are some document-wide settings and extra LaTeX packages that you may
%   want to use for your resume. Here, you can adjust:
%     -   Margin Sizes
%     -   Indentation
%     -   Hyperlink Appearance
%
%   You will also use this section to put your NAME and CONTACT INFORMATION. This
%   stuff is very very important and I would highly recommend keeping the format
%   fairly similar. This is based on advice I've received from our alumni and other
%   job recruiters.
%
%   A NOTE ON FORMATTING: Your resume should NEVER EVER EVER be more than ONE PAGE.
%             
%             Muliple pages is best left to a CV, and unless you are
%             a highly-qualified individual (read: decades of experience
%             in your field) you really should leave it to ONE PAGE.
%
%----------------------------------------------------------------------------------------

% Use the custom resume.cls style
\documentclass{content/resume/resume} 

% Document margins
\usepackage[left=0.75in,  top=0.4in,  right=0.75in,   bottom=0.3in]{geometry} 

% Stuff for setting default indentation for new paragraphs
\newcommand{\tab}[1]{\hspace{.2667\textwidth}\rlap{#1}}
\newcommand{\itab}[1]{\hspace{0em}\rlap{#1}}

% Package for proper item listing
\usepackage{enumitem}

% Package for... changing pages? I'm not too certain, and I'm not going to look it up right now.
\usepackage{changepage}

% Package for another thing I don't know about and will not look up right now.
\usepackage{parskip}

% Package for multiple columns! I'm not even sure what this is being used for. :)
\usepackage{multicol}

% Package to use *fancy* accented letters!
\usepackage[utf8]{inputenc}
\usepackage[T1]{fontenc}

%-----------------------------------------
%
%   NOTE:
%   
%   I personally like having hyperlinks to the neat stuff that I do in my resume! Especially
%   if you do not have a portfolio website, it's a fun way to directly connect digital viewers
%   of your resume to the stuff that you've done.
%
%   NOT EVERYBODY FEELS THIS WAY. It is up to your discretion whether or not you use
%   hyperlinks.
%
%   My advice, though:
%     - Keep hyperlinks set to a dark, neutral color. This will make them non-abrasive
%     compared to the black of the rest of your text, AND will make them appear normal
%     should anybody print out your resume.
%
%-----------------------------------------
\usepackage[pdfnewwindow]{hyperref}
\hypersetup{
  colorlinks=true,  % Flag for whether links should be a different color
  linkcolor=blue,   % Color of in-page links (not used)
  filecolor=magenta,  % Color of local filesystem links (not used)
  urlcolor=[rgb]{0,0,1},    % Color of web links (DEFINITELY USED!)
}

%------------------------------------------
%
%   CONTACT INFO:
%
%   DEFINITELY INCLUDE:
%     - Name
%     - Email
%     - Phone #
%
%   PROBABLY INCLUDE:
%     - ZIP Code
%     - City
%     - LinkedIn
%     - Personal Website (such as GitHub, or an actual portfolio website
%               for all of you tryhards out there.)
%
%------------------------------------------
\name{Spencer Elkington} 
\address{
  % NOTE: I include some fancy stuff in the link to my email that will automatically fill certain fields. CHANGE THIS.
  %
  % ORDER FOR LINKS: \href{ACTUAL LINK}{TEXT FOR LINK}. I choose to make the text the same as the URL, minus HTTPS://
  %
  \href{mailto:spelkington+resume@gmail.com?subject=Regarding\%20a\%20potential\%20work\%20opportunity&body=Hi,\%20Spencer!}
     {Email} \\ 
  \href{https://spelkington.github.io}{Website} \\
  \href{https://linkedin.com/in/spelkington}{LinkedIn} \\
  \href{https://github.com/spelkington}{GitHub}
}  
\address{Salt Lake City, UT}

\begin{document}

%---------------------------------------------------------------------------------------
%
%  SUMMARY
%
%   If you are so inclined, you can put a summary for yourself. MAKE IT BRIEF, like
%   the tagline of a movie. A longer summary is GREAT for your LinkedIn, but the space
%   on a resume is usually better suited for things like education and experiences.
%
%   Depending on when I want it, I will comment/uncomment this section.
%
%----------------------------------------------------------------------------------------

\begin{center}
 \vspace{-1em}
 {\em
%  Engineering solutions for market and organizational friction points with reliable design and automation.
%  Uncovering stories about markets and organizations with advanced analysis and intuitive explanations.
%  Finding exciting trends \& insights in data to solve
 }
 \vspace{-10pt}
\end{center}

%----------------------------------------------------------------------------------------
%
%  EDUCATION SECTION
%
%   For a student, this should be the FIRST SECTION on your resume! It's the
%   signal that lets people know - you're going to be graduated candidate!
%
%   DEFINITELY HAVE:
%     -   Your University
%     -   Your Major(s), Minors, and Certificates (if applicable)
%     -   Your graduation date. If you do not know, USE YOUR EXPECTED GRADUATION DATE
%
%----------------------------------------------------------------------------------------

\begin{rSection}{Education}

% TODO: Edit resume.cls to handle formatting better.
{\bf University of Utah} \hfill {\em August 2022}
\vspace{2pt}
  \\ Bachelor of Science | \href{https://eccles.utah.edu/programs/undergraduate/academics/majors/qamo/}{Quantitative Analysis of Markets \& Organizations}
  \\ Minor | \href{https://github.com/search?o=desc&q=user\%3ASpelkington&s=updated&type=Repositories}{Computer Science}

%----------------------------------------------------------------------------------------
%
%   RELEVANT TOPICS
%
%   The following is subject to debate!
%
%   I, the mighty and benevolent creator of this template, like to put my technical
%   skills under the Education section. As a student, it feels applicable.
%
%   It is well within reason to split this into a separate skills section. People
%   recommend it all the time. Just for them, here's two magic line of code to make it
%   happen:
%
%     \end{rSection}            % End education
%     \begin{rSection}{Technical Skills}  % Start Technical Skills
%
%   There! Magical!
%   
%   This section is, if we're being honest, mostly for the benefit of the bots. Recruiters will
%   sometimes use an Applicant Tracking System (ATS) to organize resumes. These systems are
%   built to compare your resume to sets of keywords and, if you don't match the keywords
%   well enough, then into the digital abyss your resume goes.
%
%   The system below is for your benefit! It allows you to organize and hot-swap keywords
%   that apply to you. Uncomment the keywords that apply best to the job you are applying
%   for.
%
%   To verify you've covered the right keywords, I highly recommend using jobscan.co.
%   It allows you to check your resume against the job description, to make sure you
%   have as many of the keywords the ATS may be looking for as possible.
%
%   BE AWARE: WHEN YOU ARE COMMENTING/UNCOMMENTING ITEMS, DO NOT LEAVE TRAILING COMMAS.
%
%----------------------------------------------------------------------------------------

%------------------------------------------
%
%   SKILLS/RELEVANT COURSEWORK
%     - Class topics (DO NOT PUT CLASS CODES HERE.)
%     - Independent study subjects
%     - Soft skills that you happen to be good at.
%
%------------------------------------------
{\bf Key Skills:}
\vspace{-1.83em}

\begin{adjustwidth}{6em}{0pt}
  \href{https://spelkington.github.io/?search=Software}{Software Dev} |
  \href{https://spelkington.github.io/?search=Games}{Game Dev} |
  \href{https://spelkington.github.io/?search=Economics}{Economics} |
  \href{https://spelkington.github.io/?search=Data}{Data} | 
  \href{https://spelkington.github.io/?search=DevOps}{DevOps} |
  \href{https://spelkington.github.io/?search=Visualization}{Visualizations} |
  \href{https://spelkington.github.io/?search=Presentation}{Presentations}
  
\end{adjustwidth}


%----------------------------------------------------------------------------------------
%
%  TECHNOLOGIES
%
%   This is a place to list of software/hardware that you are proficient in. Stuff like
%   a preferred OS, software suites like Adobe, Office, or AutoDesk, common programs
%   like Git, development platforms like Unity, Unreal, Qt, etc.
%
%----------------------------------------------------------------------------------------
\vspace{-3pt}
{\bf Software:}
\vspace{-1.83em}
\begin{adjustwidth}{6em}{0pt}
  \href{https://spelkington.github.io/?search=Spark}{Apache Spark} | 
  \href{https://spelkington.github.io/?search=GitHub}{GitHub CI/CD} | 
  \href{https://spelkington.github.io/?search=Databricks}{Databricks} | 
  EC2 | 
  Snowflake | 
  Tableau

\end{adjustwidth}

\end{rSection}

%----------------------------------------------------------------------------------------
%
%  LANGUAGES
%   
%   Section for programming languages.
%
%   Do NOT put a language here unless you would be comfortable white-boarding
%   or performing basic coding challenges in that language!
%
%   Additionally, I like to denote languages I am most comfortable with
%   by adding a (preferred) tag next to it.
%
%----------------------------------------------------------------------------------------
\vspace{-0.4em}
{\bf Languages:}
\vspace{-1.83em}
\begin{adjustwidth}{6em}{0pt}
  \href{https://spelkington.github.io/?search=Python}{Python} |
  \href{https://spelkington.github.io/?search=TypeScript}{TypeScript} | 
  \href{https://spelkington.github.io/?search=Lua}{Lua} | 
  \href{https://spelkington.github.io/?search=C++}{C++} |
  \href{https://spelkington.github.io/?search=LaTeX}{LaTeX} |
%  Scala |
  SQL |
%  Java |
  C\# |
  Bash/Shell
\end{adjustwidth}

\begin{rSection}{Experience}

  \href{https://www.linkedin.com/posts/m-science-llc_how-m-science-uses-databricks-structured-activity-6953752015013363713-EOTN/}{\bf Software Engineer, DataOps} | {\em \href{https://mscience.com}{M Science} \hfill June 2022 - Feburary 2023}
  \vspace{-6pt}
  \begin{itemize}[nosep]
    \item Lead implementation of {\bf Spark/AWS} optimizations, resulting in seven-figure compute cost reductions
    \item Construct optimized and durable ETL processes for \href{https://mscience.com/blog/?topics=223%2C223}{high-demand video game industry analysis}
    \item Plan \& construct unified DataOps infrastructure libraries to streamline financial research operations
    \item Construct DataOps {\bf CI/CD} pipelines for end-to-end {\bf Python/SQL} ETL dev lifecycles
%    \item Build \& present {\bf Tableau} dashboards for pipeline performance analytics \& business cost insights
%    \item Investigate \& implement {\bf AWS} and {\bf Spark} optimizations to reduce ETL job costs by as much as 90\%
%    \item Create infrastructure profiling frameworks to assess {\bf AWS} compute cost inefficiency \& design solutions
%    \item Recruit \& train new Analysts, Engineers \& Project Managers to grow site engineering team by \~85\%
%    \item Write \& document training material to refine analyst skills in pipeline development \& technical stack
%    \item Develop fast \& scalable {\bf Python/Spark} ETL pipelines for large-scale economic data sources
%    \item Fine-tune parameters for mission-critical economic data categorization pipelines
%    \item Build \& present {\bf Tableau} dashboards to convey health of pipeline KPI metrics
  \end{itemize}
  
  \href{https://mscience.com/}{\bf Senior Data Analyst} | {\em \href{https://mscience.com}{M Science} \hfill June 2021 - May 2022}
  \vspace{-6pt}
  \begin{itemize}[nosep]
%    \item Plan \& construct unified DataOps infrastructure to streamline financial research operations
%    \item Construct {\bf AWS CI/CD} DevOps pipelines for end-to-end {\bf Python/Spark} ETL process lifecycles
%    \item Investigate \& implement {\bf AWS} and {\bf Spark} optimizations to reduce ETL job costs by as much as 90\%
%    \item Create infrastructure profiling frameworks to assess {\bf AWS} compute cost inefficiency \& design solutions
%    \item Write \& document training material to refine analyst skills in pipeline development \& technical stack
    \item Developed fast \& scalable {\bf Python/Spark} ETL pipelines for petabyte-scale economic data sources
    \item Architected internal software library for accurate \& efficient analysis modules used across all research
    \item Built \& presented {\bf Tableau} dashboards for pipeline performance analytics \& business cost insights
    \item Fine-tuned parameters for mission-critical economic data categorization pipelines
%    \item Build \& present {\bf Tableau} dashboards to convey health of pipeline KPI metrics
  \end{itemize}

  \href{https://wasatchglobal.com/}{\bf Quant Research Intern} | {\em Wasatch Global Investors}, {\em \$31B AUM \hfill Jan 2020 - May 2021}
  \vspace{-6pt}
  \begin{itemize}[nosep]
    \item Researched portfolio allocation models to fine-tune allocation strategy across varied investment styles
    \item Developed {\bf Python/SQL} pipeline infrastructure to automate and backtest financial data analyses
    \item Designed {\bf Tableau} dashboards to monitor portfolio health \& risk throughout pandemic markets
%    \item Created experiments in {\bf Python} to adapt network and spectrum analyses to detect asset alpha signals
%    \item Conducted research into market hypotheses through bespoke model simulations
  \end{itemize}

  \href{https://slateci.io/}{\bf Networking Research Intern} | {\em Center for High-Performance Computing \hfill Mar 2019 - Jan 2020}
  \vspace{-6pt}
  \begin{itemize}[nosep]
    \item Built a {\bf Kubernetes/Docker} platform to simplify large-scale distributed scientific app deployments
    \item Constructed \& wrote project documentation site in {\bf React.js} to polish appearance for NSF grants
    \item Researched the use of {\bf Foreman} build/deploy systems to remotely structure new server cluster pools
  \end{itemize}
  
%  \href{https://www.mathnasium.com/bountiful}{\bf Center Director} | {\em Mathnasium \hfill Apr 2018 - Nov 2018}
%  \vspace{-6pt}
%  \begin{itemize}[nosep]
%    \item Directed the strategy and operations of an all-ages math tutoring center with 80+ enrolled students
%    \item Led a team of a 12+ math instructors \& worked to develop instructors' presentation \& teaching skills
%    \item Analyzed student assessment and progression data to curate \& teach individualized learning plans
%    \item Taught K-12 core and supplemental curriculum to students across varying skill levels and backgrounds
%    \item Developed creative and intuitive teaching methods to cover a range of students learning methods
%  \end{itemize}
  
%  \href{https://www.cs.utah.edu/~dejohnso/GREAT/}{\bf Programming Instructor}, {\em University of Utah \hfill May 2018 - July 2018}
%  \vspace{-6pt}
%  \begin{itemize}[nosep]
%    \item Collaborated with a small team to design and teach a two-month computer programming curriculum
%    \item Implemented data structures/algorithms, such as recursive sorts and Voronoi partitioning, in {\bf Scratch}
%  \end{itemize}

%  {\bf Housing Ambassador}, {\em University of Utah HRE \hfill June 2016 - March 2017}
%  \vspace{-6pt}
%  \begin{itemize}[nosep]
%    \item Provided tours of campus housing options to prospective students and their families
%    \item Created an {\bf Excel} process for mass e-mail processing to reduce time required to execute mailing contact lists
%  \end{itemize}
  
%  {\bf Medical Device Assembler}, {\em Bard Access Systems \hfill June 2016 - August 2016}
%  \vspace{-6pt}
%  \begin{itemize}[nosep]
%    \item Operated silicon injection machinery to produce medical device components
%    \item Followed strict medical assembly standard operating procedures
%  \end{itemize}
  
\end{rSection}

\begin{rSection}{Projects}
  
%  \href{https://spelkington.github.io/Job-Search-Hell/}{\bf Live Resume Continuous Integration Pipeline} \hfill {\em 2021}
%  \vspace{-6pt}
%  \begin{itemize}[nosep]
%    \item Designed a continuous integration pipeline to host dynamic copies of resumes via {\bf GitHub Actions}
%    \item Collaborated with university educational groups to teach pipeline implementation to undergraduates
%  \end{itemize}
  
  \href{https://databricks.com/blog/2022/07/14/using-spark-structured-streaming-to-scale-your-analytics.html}{{\bf Using Spark Structured Streaming to Scale Your Analytics} | {\em Databricks Engineering}} \hfill {\em June 2022}
  \vspace{-6pt}
  \begin{itemize}[nosep]
    \item Guest-authored engineering blog post about streaming-based ETL process cost optimizations
    \item Created \href{https://www.databricks.com/blog/2022/07/14/using-spark-structured-streaming-to-scale-your-analytics.html#:~:text=With\%20your\%20new\%20aggregated\%20data\%2C\%20you\%20can\%20throw\%20together\%20a\%20nice\%20visualization\%20to\%20do...\%20business\%20things.}{informative doodles} for maximum information delivery in a minimally professional form factor
  \end{itemize}

  \href{https://spelkington.github.io/?search=Roblox}{{\bf Independent Game Development} | {\em ROBLOX}} \hfill {\em May 2018 - June 2022}
  \vspace{-6pt}
  \begin{itemize}[nosep]
    \item Balanced freelance, contract \& hobby \href{https://github.com/search?p=1&q=user\%3ASpelkington+extension\%3Ats+extension\%3Atsx+extension\%3Alua&type=Code}{\bf Lua/TypeScript} game development \& game data analytics
    \item Remastered \& fully refactored a popular legacy lifestyle/sim game with 8 million unique plays
    \item Created a {\bf Google Cloud} integration for the Studio Game Engine to aggregate \& analyze play metrics
%    \item Engineered project to patch security vulnerabilities and emphasize project maintenance and scalability
%    \item Managed series of contract work jobs to create similar design features for varied development projects
  \end{itemize}

  \href{https://github.com/UtahTriangle/pointypal#pointypal-zoom-school-done-better}{\bf PointyPal: A Better Online Campus} \hfill {\em Aug 2020 - Dec 2021}
  \vspace{-6pt}
  \begin{itemize}[nosep]
    \item Built a class management app to provide students a better online experience through COVID-19
    \item Created and moderated a virtual campus for 600$+$ students and opened source for use at 4 universities
%    \item Conducted A/B testing to polish user experiences, resulting in peak growth rates of 100 users/mo
  \end{itemize}

%  \href{https://spelkington.github.io}{\bf CoinPal: Trust Your Friends With Your Savings!} \hfill {\em 2020}
%  \vspace{-6pt}
%  \begin{itemize}[nosep]
%    \item Created a {\bf Python} application to allow group chats to jointly manage a cryptocurrency portfolio
%    \item Implemented a custom API to allow secure \& limited interaction between voting clients and app server 
%  \end{itemize}

%  \href{https://github.com/University-of-Utah-CS3505/a8-an-educational-app-f18-ChaoticEvan}{\bf Lemonomics} \hfill {\em 2019}
%  \vspace{-6pt}
%  \begin{itemize}[nosep]
%    \item Developed a {\bf C++/Qt} educational video game for teaching business strategy and economics concepts
%    \item Ran {\bf SCRUM} development cycle with a team of seven to delegate and monitor task completion
%    \item Engineered custom upgrade, game state, and economic model systems to make development intuitive
%  \end{itemize}
  
  \href{https://devpost.com/software/beethoven-t9ud86}{\bf Beethoven} | {\em HackTheU 2019, 2nd Place out of 30 teams
  \hfill Aug 2019}
  \vspace{-6pt}
  \begin{itemize}[nosep]
    \item Designed a closed captioning and audio transcription service for deaf and hard-of-hearing students
    \item Built a peer-to-peer text \& audio streaming {\bf TypeScript} application stack using {\bf Node.js \& React}
  \end{itemize}

  
%  \href{https://www.linkedin.com/feed/update/urn:li:activity:6603722406240628736/}{\bf LED Music Visualizer} \hfill {\em 2018 - 2020}
%  \vspace{-6pt}
%  \begin{itemize}[nosep]
%    \item Created a {\bf C++} and {\bf Python} system for real-time music data analysis and visualizations
%    \item Designed a {\bf Python} music visualization tool for prototyping analysis \& visualization algorithms
%  \end{itemize}
  
%  {\bf Google Assistant Transit Tracker}
%  \hfill {\em October 2018}
%  \vspace{-6pt}
%  \begin{itemize}[nosep]
%    \item Created a Google DialogFlow application to retrieve local bus schedules via Google Assistant
%    \item Parsed and converted Google DialogFlow voice commands to {\bf SQL} queries of UTA schedule databases
%    \item Deployed application back-end to {\bf Firebase} in order to run all back-end code on the cloud
%  \end{itemize}

\end{rSection}

% \begin{rSection}{Leadership}
%  
%   \href{https://github.com/UtahTriangle/Laws}{\bf VP of Education $\rightarrow$ \bf President} | {\em Utah Chapter of Triangle Engineering \hfill Aug 2019 - May 2021}
%   \vspace{-6pt}
%   \begin{itemize}[nosep]
%     \item Created online infrastructure to balance member needs and community safety during COVID-19
%     \item Designed a peer-teaching curriculum to lead members to develop members' 
% %    \item Reconstituted chapter and passed down a 3-year plan to ensure future organizational stability
% %    \item Overhauled chapter functions to accommodate a fully-online environment during the 2020-2021 terms
% %    \item Redesigned governing organization to provide a better environment for org growth \& self-governance
%     
%   \end{itemize}
% 
% %  {\bf Founding Advisor}, {\em Utah Phi Sigma Rho Interest Group \hfill June 2020 - May 2021}
% %  \vspace{-6pt}
% %   \begin{itemize}[nosep]
% %    \item Founded and advised a university interest group for a national women's STEM organization
% %    \item Co-wrote and proofed petitions for recognition from the University and National Headquarters
% %    \item Assisted development in a parallel online campus for women in fields of engineering and sciences
% %  \end{itemize}
%   
% %  \href{https://stem.utah.gov/students/utah-stem-ambassador-program/}{\bf STEM Ambassador,} {\em Utah STEM Action Center \hfill April 2019 - Current}
% %  \vspace{-6pt}
% %  \begin{itemize}[nosep]
% %    \item Inform educators of STEM opportunities and engage students with science and technology demonstrations
% %  \end{itemize}
% 
%   \href{http://content.csbs.utah.edu/~rogers/ant5221/lab/manual.pdf}{\bf Genomic Data Science Tutoring} | {\em University of Utah \hfill Aug 2018 - Dec 2018}
%   \vspace{-6pt}
%   \begin{itemize}[nosep]
%     \item Organized \& lead a free {\bf Python} tutoring group for a graduate genetic anthropology course
%     \item Utilized stochastic learning frameworks to refine students' knowledge of genetic systems \& data analysis
%   \end{itemize}
% 
% \end{rSection}

%\vspace{1.2em}
%\begin{center}
%%  {\em References available by request} \\
%%  {\em \href{https://spelkington.github.io/Elkington_Full_Resume.pdf}{Full Resumé}}
%\end{center}

\end{document}

\hypersetup{
  pdftitle={Spencer Elkington Resume},
  pdfsubject={Resume for Software Engineering, Quantitative Analytics, and General Tomfoolery},
  pdfauthor={Spencer Elkington},
  pdfkeywords={
  
  }

}