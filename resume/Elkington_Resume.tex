%----------------------------------------------------------------------------------------
%   
%   SUPER COOL RESUME TEMPLATE
%
%   Resume Template by Spencer Elkington, for the
%   purposes of the Triangle Member Education Program.
%
%   Welcome to your new resume! I'll be annotating how to apply the concepts
%   from our resume workshop below.
%   
%   Feel free to delete this top bit if you ever post the source online - take
%   credit for all of this amazing work, if you really want!
%
%   SOME BASICS:
%   
%   Anything with a '%' (percent sign) before it is COMMENTED CODE. This will keep
%   it from appearing in the final LaTeX render. I use it for two reasons:
%
%       1. Making comments like these.
%       2. Omitting things from my resume that I don't want NOW, but may want LATER.
%
%   Things will come and go from your resume - some experiences you have may not
%   feel applicable to a job you're currently applying for - DO NOT DELETE THEM.
%   You never know when experience may become relevant again!
%
%   COMMENT IT OUT USING '%', because then you can whip out your experience again
%   when you need it!
%
%   Also, sometimes it's a cool confidence booster to un-comment everything and see
%   all the neat stuff you've done so far in your professional life :)
%
%----------------------------------------------------------------------------------------


%----------------------------------------------------------------------------------------
%
%	PACKAGES AND OTHER DOCUMENT CONFIGURATIONS
%
%   These are some document-wide settings and extra LaTeX packages that you may
%   want to use for your resume. Here, you can adjust:
%       -   Margin Sizes
%       -   Indentation
%       -   Hyperlink Appearance
%
%   You will also use this section to put your NAME and CONTACT INFORMATION. This
%   stuff is very very important and I would highly recommend keeping the format
%   fairly similar. This is based on advice I've received from our alumni and other
%   job recruiters.
%
%   A NOTE ON FORMATTING: Your resume should NEVER EVER EVER be more than ONE PAGE.
%                         
%                         Muliple pages is best left to a CV, and unless you are
%                         a highly-qualified individual (read: decades of experience
%                         in your field) you really should leave it to ONE PAGE.
%
%----------------------------------------------------------------------------------------

% Use the custom resume.cls style
\documentclass{resume/resume} 

% Document margins
\usepackage[left=0.75in,    top=0.4in,  right=0.75in,   bottom=0.3in]{geometry} 

% Stuff for setting default indentation for new paragraphs
\newcommand{\tab}[1]{\hspace{.2667\textwidth}\rlap{#1}}
\newcommand{\itab}[1]{\hspace{0em}\rlap{#1}}

% Package for proper item listing
\usepackage{enumitem}

% Package for... changing pages? I'm not too certain, and I'm not going to look it up right now.
\usepackage{changepage}

% Package for another thing I don't know about and will not look up right now.
\usepackage{parskip}

% Package for multiple columns! I'm not even sure what this is being used for. :)
\usepackage{multicol}

%-----------------------------------------
%
%   NOTE:
%   
%   I personally like having hyperlinks to the neat stuff that I do in my resume! Especially
%   if you do not have a portfolio website, it's a fun way to directly connect digital viewers
%   of your resume to the stuff that you've done.
%
%   NOT EVERYBODY FEELS THIS WAY. It is up to your discretion whether or not you use
%   hyperlinks.
%
%   My advice, though:
%       - Keep hyperlinks set to a dark, neutral color. This will make them non-abrasive
%         compared to the black of the rest of your text, AND will make them appear normal
%         should anybody print out your resume.
%
%-----------------------------------------
\usepackage{hyperref}
\hypersetup{
    colorlinks=true,    % Flag for whether links should be a different color
    linkcolor=blue,     % Color of in-page links (not used)
    filecolor=magenta,  % Color of local filesystem links (not used)
    urlcolor=[rgb]{0,0,1},      % Color of web links (DEFINITELY USED!)
}

%------------------------------------------
%
%   CONTACT INFO:
%
%   DEFINITELY INCLUDE:
%       - Name
%       - Email
%       - Phone #
%
%   PROBABLY INCLUDE:
%       - ZIP Code
%       - City
%       - LinkedIn
%       - Personal Website (such as GitHub, or an actual portfolio website
%                           for all of you tryhards out there.)
%
%------------------------------------------
\name{Spencer Elkington} 
\address{Salt Lake City, UT 84102 \\
         (775) 388-7033}
\address{
    %
    % NOTE: I include some fancy stuff in the link to my email that will automatically fill certain fields. CHANGE THIS.
    %
    % ORDER FOR LINKS: \href{ACTUAL LINK}{TEXT FOR LINK}. I choose to make the text the same as the URL, minus HTTPS://
    %
    \href{mailto:spelkington@gmail.com?subject=Regarding\%20a\%20potential\%20work\%20opportunity&body=Hi,\%20Spencer!}
         {spelkington@gmail.com} \\ 
    \href{https://linkedin.com/in/spelkington}{linkedin.com/in/spelkington} \\ 
    \href{https://github.com/spelkington}{github.com/spelkington}
}    

\begin{document}

%---------------------------------------------------------------------------------------
%
%	SUMMARY
%
%   If you are so inclined, you can put a summary for yourself. MAKE IT BRIEF, like
%   the tagline of a movie. A longer summary is GREAT for your LinkedIn, but the space
%   on a resume is usually better suited for things like education and experiences.
%
%   Depending on when I want it, I will comment/uncomment this section.
%
%----------------------------------------------------------------------------------------

\begin{center}
 \vspace{-1em}
 {\em
%    Engineering solutions for market and organizational friction points with reliable design and automation.
%    Uncovering stories about markets and organizations with advanced analysis and intuitive explanations.
%    Exploring the relationships in data to discover exciting trends in the world around us.
 }
 \vspace{-10pt}
\end{center}

%----------------------------------------------------------------------------------------
%
%	EDUCATION SECTION
%
%   For a student, this should be the FIRST SECTION on your resume! It's the
%   signal that lets people know - you're going to be graduated candidate!
%
%   DEFINITELY HAVE:
%       -   Your University
%       -   Your Major(s), Minors, and Certificates (if applicable)
%       -   Your graduation date. If you do not know, USE YOUR EXPECTED GRADUATION DATE
%
%----------------------------------------------------------------------------------------

\begin{rSection}{Education}

% TODO: Edit resume.cls to handle formatting better.
{\bf University of Utah} \hfill {\em December 2021}
\vspace{2pt}
\emph{
    \\ B.S. Quantitative Analysis of Markets \& Organizations
    \\ Minor Computer Science
}

%----------------------------------------------------------------------------------------
%
%   RELEVANT TOPICS
%
%   The following is subject to debate!
%
%   I, the mighty and benevolent creator of this template, like to put my technical
%   skills under the Education section. As a student, it feels applicable.
%
%   It is well within reason to split this into a separate skills section. People
%   recommend it all the time. Just for them, here's two magic line of code to make it
%   happen:
%
%       \end{rSection}                      % End education
%       \begin{rSection}{Technical Skills}  % Start Technical Skills
%
%   There! Magical!
%   
%   This section is, if we're being honest, mostly for the benefit of the bots. Recruiters will
%   sometimes use an Applicant Tracking System (ATS) to organize resumes. These systems are
%   built to compare your resume to sets of keywords and, if you don't match the keywords
%   well enough, then into the digital abyss your resume goes.
%
%   The system below is for your benefit! It allows you to organize and hot-swap keywords
%   that apply to you. Uncomment the keywords that apply best to the job you are applying
%   for.
%
%   To verify you've covered the right keywords, I highly recommend using jobscan.co.
%   It allows you to check your resume against the job description, to make sure you
%   have as many of the keywords the ATS may be looking for as possible.
%
%   BE AWARE: WHEN YOU ARE COMMENTING/UNCOMMENTING ITEMS, DO NOT LEAVE TRAILING COMMAS.
%
%----------------------------------------------------------------------------------------

%------------------------------------------
%
%   SKILLS/RELEVANT COURSEWORK
%       - Class topics (DO NOT PUT CLASS CODES HERE.)
%       - Independent study subjects
%       - Soft skills that you happen to be good at.
%
%------------------------------------------
{\bf Talents:}
\vspace{-1.83em}

\begin{adjustwidth}{6em}{0pt}
%    \href{https://github.com/Spelkington/mlearning}{Algorithms},
    \href{https://github.com/Spelkington/mlearning}{Machine Learning},
    \href{https://www.youtube.com/watch?v=lMFQp3wN-cg}{Economics},
    Data Science,
    Algorithms,
%    Data Structures,
    Software Dev,
%    \href{https://github.com/Spelkington/econometrics}{Econometrics},
%    Data Structures,
%    Data Analytics,
%    AI,
%    Artificial Intelligence,
%    Software Dev,
    Statistics
%    Game Dev,
%    Mentoring
%    Linear Alg.,
%    Object-Oriented Design
%    A/V Codecs,
%    Strategy
%    Financial Modeling
%    Quantitative Analytics,
%    A/B Testing,
%    Game Theory,
%    Data Mining,
%    AGILE,
%    Quantitative Research
%    Research
%    Digital Economics,
%    Communication,
%    Technical Writing
%    Model Design,
%    Distributed Computing
    
\end{adjustwidth}


%----------------------------------------------------------------------------------------
%
%	TECHNOLOGIES
%
%   This is a place to list of software/hardware that you are proficient in. Stuff like
%   a preferred OS, software suites like Adobe, Office, or AutoDesk, common programs
%   like Git, development platforms like Unity, Unreal, Qt, etc.
%
%----------------------------------------------------------------------------------------
\vspace{-3pt}
{\bf Software:}
\vspace{-1.83em}
\begin{adjustwidth}{6em}{0pt}
    \href{https://github.com/search?q=user\%3ASpelkington+user\%3AUtahTriangle+extension\%3Aipynb&type=Code}{Jupyter},
    Linux,
%    Kubernetes,
    Docker,
    Tableau,
%    Ubuntu,
%    CentOS,
%    RHES,
%    Windows,
%    Bash,
    Git,
%    Qt,
%    React,
    Node.js,
    MySQL,
%    Anaconda,
%    FFMPEG,
%    Arduino,
%    Travis,
%    STATA,
%    Audacity,
    Excel/VBA,
    Scikit-Learn
%    FactSet

\end{adjustwidth}

\end{rSection}

%----------------------------------------------------------------------------------------
%
%	LANGUAGES
%   
%   Section for programming languages.
%
%   Do NOT put a language here unless you would be comfortable white-boarding
%   or performing basic coding challenges w/o docs in that language!
%
%   Additionally, I like to denote languages I am most comfortable with
%   by adding a (preferred) tag next to it.
%
%----------------------------------------------------------------------------------------
\vspace{-0.4em}
{\bf Languages:}
\vspace{-1.83em}
\begin{adjustwidth}{6em}{0pt}
    \href{https://github.com/search?q=user\%3ASpelkington+user\%3AUtahTriangle+extension\%3Apy+extension\%3Aipynb&type=Code&ref=advsearch&l=&l=}{Python} (preferred),
    C++,
%    Visual Basic/VBA,
    JavaScript,
    SQL,
    Java,
    C\#,
%    TypeScript
%    STATA
%    C,
    Lua
%    R
%    HTML/CSS
\end{adjustwidth}

%----------------------------------------------------------------------------------------
%
%	WORK EXPERIENCE SECTION
%
%   This is the area where you'll put positions you've held at places you've worked for!
%
%   Things like company and university jobs should go here. Unpaid research positions can
%   also go here.
%
%   DEFINITELY HAVE:
%       -   Job Title
%       -   Location
%       -   Time Period (Start - End, or Start - Current if you still work there.)
%       -   1 - 3 XYZ Bullets describing your work and responsibilities
%
%   XYZ FORMATTING:
%
%   Performed [X], resulting in [Y], using [Z].
%
%   X: What you did
%   Y: Why you did it
%   Z: How you did it
%
%   X and Z are the most important bits - Y is nice if you have quantifiable evidence
%   of the result of your action.
%
%   Remember, you can comment out bits that don't apply to the position you're tailoring
%   your resume for but DO NOT DELETE THEM. Good XYZ bullets are hard to come by and you
%   never know when you might need them again.
%
%----------------------------------------------------------------------------------------

\begin{rSection}{Experience}

    {\bf Quantitative Research Intern}, {\em Wasatch Global Investors - Remote \hfill Jan 2020 - May 2021}
    \vspace{-6pt}
    \begin{itemize}[nosep]
        \item Designed statistical allocation models to market and boost performance of investment portfolios
%        \item Conducted research into market hypotheses through bespoke model simulations
        \item Created experiments in {\bf Python} to adapt network and spectrum analyses to financial forecasting
        \item Developed a {\bf Python/SQL} data pipeline to ease and automate collection of financial data
        \item Designed visualizations and dashboards in {\bf Tableau} to tell intuitive stories with data
    \end{itemize}

    {\bf \href{https://slateci.io/}{SLATE Dev Intern}}, {\em Utah Center for High-Performance Computing \hfill Mar 2019 - Feb 2020}
    \vspace{-6pt}
    \begin{itemize}[nosep]
        \item Built a {\bf Kubernetes/Docker} platform to simplify deploy of science apps on cloud edge systems
        \item Constructed project documentation site in {\bf React.js} and monitored site and project metrics
        \item Researched the use of {\bf Foreman} provisioning software to remotely structure new server clusters
    \end{itemize}
    
    {\bf Center Director}, {\em Mathnasium of Utah \hfill Apr 2018 - Nov 2018}
    \vspace{-6pt}
    \begin{itemize}[nosep]
        \item Directed the strategy and operations of a K-12 math tutoring center with 80 enrolled students
%        \item Taught K-12 core and supplemental curriculum to students across varying skill levels and backgrounds
%        \item Developed creative and intuitive teaching methods to cover a range of students learning methods
        \item Led a team a dozen skilled math instructors in refining teaching and center presentation practices
%        \item Analyzed student testing and progression data to curate \& teach individualized learning plans
    \end{itemize}
    
%    {\bf Programming Instructor}, {\em University of Utah Summer Camps \hfill May 2018 - July 2018}
%    \vspace{-6pt}
%    \begin{itemize}[nosep]
%        \item Collaborated with a small team to design and teach a two-month computer programming curriculum
%        \item Implemented data structures/algorithms, such as recursive sorts and Voronoi partitioning, in {\bf Scratch}
%    \end{itemize}

%    {\bf Housing Ambassador}, {\em University of Utah HRE \hfill June 2016 - March 2017}
%    \vspace{-6pt}
%    \begin{itemize}[nosep]
%        \item Provided tours of campus housing options to prospective students and their families
%        \item Created an {\bf Excel} process for mass e-mail processing to reduce time required to execute mailing contact lists
%    \end{itemize}
%    
%    {\bf Medical Device Assembler}, {\em Bard Access Systems \hfill June 2016 - August 2016}
%    \vspace{-6pt}
%    \begin{itemize}[nosep]
%        \item Operated silicon injection machinery to produce medical device components
%        \item Followed strict medical assembly standard operating procedures
%    \end{itemize}
    
\end{rSection}

%-----------------------------------------------------------------------------------------------
%
%   PROJECTS
%
%   This is a place to put big, cool things that you've worked on and have something to show for.  
%
%   Here, I personally like to put three things:
%       - Personal Projects
%       - Open-Ended School Projects

%       - Demonstrable Work Projects
%
%   This functions as a micro-portfolio of your work. If you feel comfortable showing it off,
%   put it here.
%
%   I also highly recommend putting links for projects, wherever applicable.
%
%-----------------------------------------------------------------------------------------------
\begin{rSection}{Projects}

    \href{https://github.com/UtahTriangle/pointypal}{\bf PointyPal: A Better Online Campus}
    \vspace{-6pt}
    \begin{itemize}[nosep]
        \item Built a class management application to provide students a better online experience during COVID-19
        \item Created and moderated a virtual campus for 450+ students to test application prior to opening source 
        \item Conducted A/B testing to polish user experiences, resulting in peak growth rates of 100 users/mo
    \end{itemize}

%    \href{https://github.com/University-of-Utah-CS3505/a8-an-educational-app-f18-ChaoticEvan}{\bf Lemonomics}
%    \vspace{-6pt}
%    \begin{itemize}[nosep]
%        \item Developed a {\bf C++/Qt} educational video game for teaching business strategy and economics concepts
%        \item Ran {\bf SCRUM} development cycle with a team of seven to delegate and monitor task completion
%        \item Engineered custom upgrade, game state, and economic model systems to make development intuitive
%    \end{itemize}
    
    \href{https://devpost.com/software/beethoven-t9ud86}{\bf Beethoven}, {\em 2nd Place out of 30 teams
    \hfill HackTheU 2019}
    \vspace{-6pt}
    \begin{itemize}[nosep]
        \item Designed a closed captioning and audio transcription service for deaf and hard-of-hearing students
        \item Built a peer-to-peer text \& audio streaming {\bf TypeScript} app using {\bf Node.js \& React}
    \end{itemize}
    
    \href{https://www.linkedin.com/feed/update/urn:li:activity:6603722406240628736/}{\bf LED Music Visualizer}
    \vspace{-6pt}
    \begin{itemize}[nosep]
      \item Created a {\bf C++} and {\bf Python} system for real-time music data analysis and visualizations
      \item Designed a {\bf Python} music visualization tool for prototyping analysis \& visualization algorithms
    \end{itemize}

%    \href{https://www.roblox.com/games/272941/Robloxaville}{\bf Robloxaville}
%    \vspace{-6pt}
%    \begin{itemize}[nosep]
%      \item Remastered a popular {\bf Lua} game on the ROBLOX platform, supporting both PC \& mobile gameplay
%      \item Engineered project to patch security flaws and emphasize project maintainability and scalability 
%    \end{itemize}
    
%    {\bf Google Assistant Transit Tracker}
%    \hfill {\em HackTheU 2018}
%    \vspace{-6pt}
%    \begin{itemize}[nosep]
%      \item Created a Google DialogFlow application to retrieve local bus schedules via Google Assistant
%      \item Parsed and converted Google DialogFlow voice commands to {\bf SQL} queries of UTA schedule databases
%      \item Deployed application back-end to {\bf Firebase} in order to run all back-end code on the cloud
%    \end{itemize}

\end{rSection}


%----------------------------------------------------------------------------------------------
%
%   Activities
%
%   This is a great place to put things that you've put time into that aren't necessarily
%   a job or paid opportunity. This includes:
%       - Organizations
%       - Clubs
%       - Meetup groups
%       - Volunteering
%       - !!!Leadership positions!!!
%
%   Be certain, though, that you've participated in big things through these! Don't just put
%   that you were a member - talk about what you did!
%
%   NOTE FOR TRIANGLE MEMBERS: This is a great place to put leadership positions,
%                              like chairs and exec roles!
%                              
%                              Because all executive positions are Vice President
%                              roles, typically "Vice President" will suffice. If
%                              you really want to get specific, you could say
%                              "VP External/Internal/Treasury/etc.", but I personally
%                              don't feel the extra flair is necessary.
%                              
%                              Additionally, THIS IS A FANTASTIC REASON TO GET
%                              HANDS-ON IN OUR CHAPTER! The perk of being part
%                              of a young organization is that you can make a
%                              massive impact for everybody, and brag about it
%                              here!
%   
%-----------------------------------------------------------------------------------------------

\begin{rSection}{Activities}

    {\bf VP of Education} $\rightarrow$ {\bf President}, {\em Utah Chapter of Triangle \hfill Apr 2019 - Current}
    \vspace{-6pt}
    \begin{itemize}[nosep]
        \item \href{https://github.com/UtahTriangle/Laws}{Reconstituted chapter} and passed down a 3-year plan to ensure future organizational stability
%        \item Pivoted chapter functions to accommodate a fully-online environment during the 2020-2021 terms
        \item \href{https://github.com/UtahTriangle/Laws/raw/main/Proposals/IGC/The\%20Independent\%20Greek\%20Council.pdf}{Redesigned \& presented} governing organization designs to provide a better environment for growth
    \end{itemize}
    
%    {\bf Founding Advisor}, {\em Utah Phi Sigma Rho Interest Group \hfill June 2020 - Current}
%    \vspace{-6pt}
%    \begin{itemize}[nosep]
%        \item Founded and advised a university interest group for a national women's STEM organization
%        \item Co-wrote and proofed petitions for recognition from the University and National Headquarters
%        \item Assisted development in a parallel online campus for women in fields of engineering and sciences
%    \end{itemize}
    
%    \href{https://stem.utah.gov/students/utah-stem-ambassador-program/}{\bf STEM Ambassador,} {\em Utah STEM Action Center \hfill April 2019 - Current}
%    \vspace{-6pt}
%    \begin{itemize}[nosep]
%        \item Inform educators of STEM opportunities and engage students with science and technology demonstrations
%    \end{itemize}
    
    {\bf Genome Analysis Tutor}, {\em University of Utah \hfill Fall Term, 2017}
    \vspace{-6pt}
    \begin{itemize}[nosep]
      \item Organized \& lead a free {\bf Python} tutoring group for a \href{http://content.csbs.utah.edu/~rogers/ant5221/lab/manual.pdf}{graduate anthropology course}
      \item Utilized stochastic learning algorithms to track genetic drift in time-series genetic datasets
    \end{itemize}

\end{rSection}

%-----------------------------------
%
%   REFERENCES
%
%   So, typically, references shouldn't go directly on your resume. If the recruiter would
%   like a reference for something on your application, they would either a) ask you directly
%   or b) scout a service like LinkedIn for that sort of thing.
%
%   Also, there have been arguments that something like "References Upon Request" is implied.
%   I personally don't see any harm in doing it - it's a very short line item. However, your
%   mileage may vary.
%
%-----------------------------------
\vspace{1.2em}
\begin{center}
    {\em References available by request} \\
    {\em \href{https://www.overleaf.com/read/dpkcngtfrygt}{Full Resume Source}}
\end{center}

\end{document}

%-----------------------------------
%
%   METADATA
%
%   This is a personal testing point - I was curious to see if ATS picks up
%   the .pdf metadata as part of the main resume or not. Results were inconclusive -
%   use at your own risk.
%  
%-----------------------------------
\hypersetup{
    pdftitle={Spencer Elkington Resume},
    pdfsubject={Resume for Software Engineering, Quantitative Analytics, and General Tomfoolery},
    pdfauthor={Spencer Elkington},
    pdfkeywords={
    
                }
}